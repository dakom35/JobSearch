\documentclass[11pt,a4paper]{moderncv}
\usepackage{pifont}
% Thèmes ModernCV
\moderncvstyle{classic} % Choisissez le style
\moderncvcolor{blue} % Choisissez la couleur
\usepackage{geometry}
\geometry
{
    left=0.3in,
    right=0.6in,
    top=0.05in,
    bottom=0.5in,
}

\usepackage{array}
\usepackage[table]{xcolor}
\usepackage{pifont}
%\pagestyle{empty} % devrait supprimer le numéro de page

% Données personnelles
\name{Alfred}{Lalanne}
\title{Développeur Systèmes Embarqués}
\address{Paris, France}
\phone[mobile]{+33~(6)~88~33~06~05}
\email{alfred.lalanne@gmail.com}
\social[linkedin][linkedin.com/in/alfred-lalanne-714b5b192/]{Mon LinkedIn}
\social[github][github.com/dakom35]{Mon GitHub}
\photo[64pt][0.4pt]{moi.jpeg}

\begin{document}

\makecvtitle

\section{Formation}
    \cventry
        {2019--2023}
        {Diplôme d'Ingénieur en Électronique}
        {ENSEA}
        {Cergy (95)}
        {}
        {Formation complète couvrant l'électronique analogique et numérique, le traitement du signal, la conception de systèmes embarqués, les systèmes de communication et la gestion de projets.}
    \cventry
        {2017--2019}
        {Classes Préparatoires aux Grandes Écoles d'Ingénieurs}
        {Lycée Brizeux}
        {Quimper (29)}
        {}
        {PCSI / PSI}   
    \cventry
        {2016--2017}
        {Classe Préparatoire aux Etudes Supérieures}
        {Lycée Naval}
        {Brest (29)}
        {}
        {Année de remise à niveau avec encadrement militaire}  

        
\section{Expérience}
\cventry
    {2024--ce jour}
    {Ingénieur en Test Électrique}
    {ICE TECH SAS}
    {Buchelay (78)}
    {}
    {
    Responsable des tests électriques des cartes à puce 
    \begin{itemize}
        \setlength{\itemindent}{1cm}
        \item Développement intégral d'un logiciel pour analyser la qualité de la production des cartes à puce
        \begin{itemize}
            \setlength{\itemindent}{1.5cm}
            \item Analyse des informations contenues dans les fichiers journaux provenant du testeur électrique
            \item Développement des algorithmes d'analyse et de représentation des données
            \item Refactorisation en une bibliothèque orientée objet (OOP)
            \item Intégration de cette bibliothèque dans une interface utilisateur conviviale
        \end{itemize}
        \item Rédaction de documentation concernant le testeur électrique
        \item Support aux opérateurs lors de la production
        \item Résolution de pannes complexes concernant le testeur électrique
        \begin{itemize}
            \setlength{\itemindent}{1.5cm}
            \item Détermination et résolution de la panne
            \item Mise en œuvre de mesures préventives
        \end{itemize}
    \end{itemize}
    }
\cventry
    {2020--2023}
    {Ingénieur Apprenti}
    {IDEMIA}
    {Osny (95)}
    {}
    {
    Développement d'algorithmes de traitement d'images thermiques en Python (4 mois) \newline
    Contribution au projet ALIX, un système visant à identifier les propriétaires de bagages perdus dans les aéroports. 
    \begin{itemize}
        \setlength{\itemindent}{1cm}
        \item Modification du cahier des charges pour le pilote industriel
        \item Modélisation 3D du système pour simuler les champs de vision des caméras avec Blender
        \item Recherche et tests du composant de détection des bagages
        \item Analyse, modification et test du driver d'éclairage LED
        \item Élaboration et vérification du schéma électrique global
        \item Optimisation du rapport signal à bruit des caméras en utilisant OpenCV pour C\texttt{++}
        \item Conception d'un banc industriel pour la calibration des caméras, optimisé pour l'industrialisation
        \begin{itemize}
            \setlength{\itemindent}{1.5cm}
            \item Conception de la structure mécanique 
            \item Définition de la distance de mise au point (mires - capteur)
            \item Développement et intégration d'algorithmes C\texttt{++} pour :
            \begin{itemize}
                \setlength{\itemindent}{2.5cm}
                \item Détection automatique des mires (ArUco)
                \item Réglage de la mise au point (calcul de MTF)
                \item Calibration de la balance des blancs 
            \end{itemize}
        \end{itemize}
    \end{itemize}
    }

    
\section{Compétences}

\cvitem
    {Langages}
    {C\texttt{++} \ding{72}\ding{72}\ding{72}, C \ding{72}\ding{72}\ding{72} , Python \ding{72}\ding{72}\ding{73} , Matlab \ding{72}\ding{72}\ding{73} , Java \ding{72}\ding{73}\ding{73} , Javascript \ding{72}\ding{73}\ding{73} }
\cvitem
    {Librairies}
    {OpenCV \ding{72}\ding{72}\ding{72} , PyQt5 \ding{72}\ding{72}\ding{72} , NumPy \ding{72}\ding{72}\ding{73} , SFML \ding{72}\ding{72}\ding{73} , Matplotlib \ding{72}\ding{72}\ding{73}, RealSense \ding{72}\ding{73}\ding{73}}
\cvitem
    {Simulation}
    {LTSpice \ding{72}\ding{72}\ding{73} , OrCAD PSpice \ding{72}\ding{72}\ding{73}, Blender \ding{72}\ding{73}\ding{73}}

\section{Langues}
\cvitemwithcomment{Français}{Langue maternelle}{}
\cvitemwithcomment{Anglais}{Maîtrise professionnelle complète}{Examen TOEIC 915/990}


\end{document}
```